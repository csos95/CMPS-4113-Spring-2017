\documentclass{scrreprt}
\usepackage{graphicx}
\graphicspath{ {images/} }
\usepackage{listings}
\usepackage{underscore}
\usepackage[bookmarks=true]{hyperref}
\usepackage{hyperref}
\usepackage{graphicx}
\usepackage[english]{babel}
\usepackage[utf8]{inputenc}
\pagenumbering{arabic}
\usepackage{fancyhdr}
\fancypagestyle{plain}{
	\fancyhf{}% Clear header/footer
	\fancyhead[L]{Software Metrics Calculation System User Manual}% Left header
	\fancyfoot[C]{\thepage}
}

\hypersetup{
	bookmarks=false,    % show bookmarks bar?
	pdftitle={Software Metrics Calculation System User Manual},    % title
	pdfauthor={Christopher Silva},                     % author
	pdfsubject={TeX and LaTeX},                        % subject of the document
	pdfkeywords={TeX, LaTeX, graphics, images}, % list of keywords
	colorlinks=true,       % false: boxed links; true: colored links
	linkcolor=blue,       % color of internal links
	citecolor=black,       % color of links to bibliography
	filecolor=black,        % color of file links
	urlcolor=purple,        % color of external links
	linktoc=page            % only page is linked
}%
\author{Christopher Silva}
\date{}
\def\myversion{1.0 }
\begin{document}
	\begin{titlepage}
		\flushright
		\LARGE{ID-10-T}
		\includegraphics[scale=0.08]{logo.png}
		\rule{16cm}{5pt}\vskip1cm
		\centering
		\Huge{Software Metrics Calculation System}\\
		\vspace{2cm}
		\Huge{USER MANUAL}\\
		\vspace{2cm}
		\LARGE{Version \myversion\\}
		\vspace{2cm}
		Prepared by\\
	    Christopher Silva\\
	    Shujing Zhang\\
		Anthony Enem\\
		Nathan Durst\\
		Da Dong\\
		\vfill
		\rule{16cm}{5pt}
	\end{titlepage}
	\pagenumbering{roman}
	\tableofcontents
%============================================================================
	\chapter{Introduction}
	\pagenumbering{arabic}
	
	\section{Overview}
	The Software Metrics Calculation System (SMCS) is a piece of software that allows the user to analyze and view various metrics about their source code. \\
	This manual will describe the features and usage of SMCS.
	
	\section{Background}
	Dr. Stringfellow was interested in a software system that would help ensure that her computer science students are writing programs that fit her specifications.
	SMCS was created by the Software Engineering group ID-10-T to fulfill this need.
	
	\section{How to use this document}
	This document is split into four chapters. \\
	Chapter 1: An introduction to SMCS. \\
	Chapter 2: Describes the purpose of SMCS and the software requirements. \\
	Chapter 3: Details how to use SMCS. \\
	Chapter 4: Describes the metrics available in SMCS. \\
	Chapter 5: Lists contact information and how to report a bug.
	
%============================================================================
	{\let\clearpage\relax \chapter{Software Overview}}
	
	\section{Software Purpose}
	The main objective of SMCS is to help first and second year computer science students become better programmers.
	This software will calculate and display metrics about the users source code such as line of code, lines of documentation, the ratio of the two, cyclical complexity, etc.
	These metrics will be used to give the user feedback on their code and to correct commonly made mistakes.

	\section{Software Requirements}
	The minimum requirements for SMCS are relatively low.\\
	Minimum Operating System: Windows XP, OS X 10.7, Linux 2.6.23\\
	Web Browser: Firefox, Chrome, Edge\\

%============================================================================
	{\let\clearpage\relax \chapter{Getting Started}}
	\section{How to Start SMCS}
	To start SMCS in standalone mode simply double click on the application. \\
	This will start a local web server on port 8080 and launch your default web browser to the SMCS code submission page. \\
	If you wish to host SMCS on a server so that it is accessible from anywhere, open config.json and change the "standalone" configuration option to false, ensure that port 8080 if forwarded correctly and then start SMCS. \\
	When not in standalone mode, SMCS will not automatically open in a browser.

	\section{User interface}
	The user interface is split into two pages, a source code submission page, and a analysis results page. \\
	
	GRAPHIC OF SUMISSION PAGE WITH NUMBERED ELEMENTS HERE \\
	
	LIST  OF WHAT THE NUMBERED UI ELEMENTS REPRESENT HERE \\
	
	GRAPGHIC OF RESULTS PAGE WITH NUMBERED ELEMENTS HERE \\
	
	LIST  OF WHAT THE NUMBERED UI ELEMENTS REPRESENT HERE \\
	
	USE CASES HERE \\
	

%============================================================================
	{\let\clearpage\relax \chapter{Available Metrics}}
	
	\section{Lines of ...}
	\begin{itemize}
		\item Lines of Code - The number of lines of code in your source.
		\item Lines of Documentation - The number of lines of documentation in your source.
		\item Ratio of LOC to LOD - It could be a problem if the ratio of documentation to code is too low.
		\item Blank Lines - The number of blank lines in you source.
		\item Total Lines - The total number of lines in your source.
	\end{itemize}
	Note: A line that contains code and documentation will be counted as both a line of code and documentation.
	
	\section{Number of ...}
	\begin{itemize}
		\item Number of Functions - The number of functions in your source.
		\item Number of Function parameters - The number of parameters in each function. If you have too many parameters in a function it could cause trouble, especially if it is used often.
		\item Methods per Class - The number of methods in each class.
		\item Lines per Function - the number of lines in each function.
	\end{itemize}
	
	\section{Cyclomatic Complexity}
	Cyclomatic Complexity is the number of linearly independent paths within a program. \\
	It is used to measure the complexity of a program. \\
	
	Example C code:
	\begin{lstlisting}
	if (A == 10) {
	    if (B > C) {
	        A = B;
	    } else {
	        A = C;
	    }
	}
	printf(A);
	printf(B);
	printf(C);
	\end{lstlisting}
	This code has a three  separate paths that it can take so it has a cyclomatic complexity of three. 

%============================================================================	
	{\let\clearpage\relax \chapter{Appendix}}
	
	\section{How to Report a Bug}
	If you find any bugs, please tell us by sending an email to ID10T.BUG@example.com\\
	Please include the following information when reporting a bug:\\
	\begin{itemize}
		\item A complete description of the problem what led to it.
		\item What operating system and web browser you are using.
		\item What error messages are displayed.
	\end{itemize}
	
	\section{Contacts}
		Contact information for ID-10-T team members.\\
		\begin{tabular}{|l|c|}
			\hline
			Member Name       & Phone Number \\ \hline
			Christopher Silva & 940-782-1234 \\ \hline
			Anthony Enem      & 940-782-2345 \\ \hline
			Nathan Durst      & 940-782-3456 \\ \hline
			Da Dong           & 940-782-4567 \\ \hline
			Shujing Zhang     & 940-782-6789 \\ \hline
		\end{tabular}
\end{document}