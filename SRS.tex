\documentclass{scrreprt}
\usepackage{listings}
\usepackage{underscore}
\usepackage[bookmarks=true]{hyperref}
\usepackage{hyperref}
\hypersetup{
	bookmarks=false,    % show bookmarks bar?
	pdftitle={Software Requirement Specification},    % title
	pdfauthor={Christopher Silva},                     % author
	pdfsubject={TeX and LaTeX},                        % subject of the document
	pdfkeywords={TeX, LaTeX, graphics, images}, % list of keywords
	colorlinks=true,       % false: boxed links; true: colored links
	linkcolor=blue,       % color of internal links
	citecolor=black,       % color of links to bibliography
	filecolor=black,        % color of file links
	urlcolor=purple,        % color of external links
	linktoc=page            % only page is linked
}%
\author{Christopher Silva}
\date{}
\def\myversion{0.1 }
\begin{document}
	\begin{titlepage}
		\flushright
		\rule{16cm}{5pt}\vskip1cm
		\Huge{SOFTWARE REQUIREMENTS\\ SPECIFICATION}\\
		\vspace{2cm}
		for\\
		\vspace{2cm}
		CMPS 4113 - Software Engineering\\
		\vspace{2cm}
		%\LARGE{\myversion\\}
		%\vspace{2cm}
		%\LARGE{Version \myversion approved\\}
		%\vspace{2cm}
		%Prepared by Christopher Silva\\
		\vfill
		\rule{16cm}{5pt}
	\end{titlepage}
	\tableofcontents
	\chapter*{Revision History}
	February 4, 2017 Initial Draft - Christopher Silva, Anthony Enem, Nathan Durst, Da Dong, and Shujing Zhang.
	\chapter{Introduction}
	Dr. Stringfellow (hereafter referred to as the client), is interested in software that will help ensure that her computer science students are writing programs that fit her specifications. This software should calculate and display metrics about the users source code such as line of code, lines of documentation, and the ratio of the two.
	\section{Purpose}
	This document details the Project Plan for the Software Metrics Calculation System (hereafter referred to as SCMS), which the Software Engineering group ID-10-T (hereafter also referred to as the team) has devised to assist in the software development process. The plan outlines the different areas of the project that must be addressed for successful development of the software. It establishes guidelines for resources that will be used in the project, and also points out additional resources that are needed. The Project Plan shows how the team is comprised and states the means of reporting. This plan addresses some of the risks involved in the project and the steps to correct those risks, if they occur. Also, quality assurance will be mentioned, and a glossary of terms used in this document is included.
	\section{Scope}
	The client wants SMCS to quickly calculate code metrics on student source code. The client currently spends an excess amount of time looking for issues that could be solved if students had software to point them out. The client would like SMCS to support C++ and Java source code. The client would like SMCS to be easily extensible in the future to allow for more types of metrics or languages.
	\section{Main Objective}
	The main objective of SMCS is to help first and second year computer science students become better programmers by giving them a tool that will point out some frequent simple mistakes that they make.
	\section{Overview of Document}
	The remainder of the document is intended to inform the client of the intended system. Hardware and software requirements, major users, both major and minor functions, constraints, and intended user interface are described.
	\chapter{Users}
	\section{Who are the Users?}
	\section{Use Cases}
	\section{Scenarios}
	\chapter{Overall Description}
	\section{Product Perspective}
	\section{User Classes and Characteristics}
	% add other chapters and sections to suit
\end{document}